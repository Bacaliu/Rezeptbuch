\documentclass[twoside=true, DIV=16, 11pt,a4paper]{scrreprt}
\usepackage{cuisine}
\usepackage[clock]{ifsym}

\input{~/Vorlagen/template}
\input{~/Vorlagen/seitenstil-reprt}
\chead*{Rezepte}
\autoren{%Simon Smolarek,
		Adrian Bacaliu
}
\renewcommand*{\recipefont}{\ttfamily\slshape}
\renewcommand*{\recipenumberfont}{\LARGE\ttfamily\itshape}
\renewcommand*{\recipetitlefont}{\LARGE\ttfamily\itshape}
%\renewcommand*{\recipeservingsfont}{\bfseries\itshape\ttfamily}
%\renewcommand*{\recipetimefont}{\bfseries\itshape\ttfamily}
%\renewcommand*{\recipestepnumberfont}{\itshape\bfseries}
%\renewcommand*{\recipequantityfont}{\ttfamily}
%\renewcommand*{\recipeunitfont}{\ttfamily\slshape}

\newcommand{\zeit}[2]{
	\ifthenelse{\equal{#1}{0}}{
		\showclock{#1}{#2} #2\,min
	}{
		\ifthenelse{\equal{#2}{0}}{
			\showclock{#1}{#2} #1\,h
		}{
			\showclock{#1}{#2} #1h~#2\,min
		}
	}
}

\begin{document}
	\tableofcontents
	\chapter{Kuchen}
	\ohead*{Kuchen}
	\addcontentsline{toc}{section}{Fantawaffeln}
	\begin{recipe}{Fantawaffeln}{4 Portionen}{\zeit{0}{15}}
		\ing[300]{g}{Mehl}
		\ing[150]{g}{Zucker}
		\ing[\fr{1}{2}]{TL}{Backpulver}
		\ing[1]{Pkg.}{Vanillezucker}
		Alle trockenen Zutaten miteinander vermengen.
		\ing[3]{}{Eier}
		\ing[175]{ml}{Öl}
		\ing[200]{ml}{Fanta}
		Dann die weiteren Zutaten hinzufügen und mit dem Mixer verrühren.
		\newstep
		Schließlich normal im Waffeleisen backen.
	\end{recipe}\vfill
	\addcontentsline{toc}{section}{Fantakuchen}
	\begin{recipe}{Fantakuchen mit Schmand und Früchten}{Backblech}{\zeit{1}{0}}
		\freeform{\Large Biscuit}
		\ing[5]{}{Eier}
		\ing[300]{g}{Zucker}
		\ing[1]{Pkg.}{Vanillezucker}
		Eier und (Vanille-)Zucker schaumig rühren.
		\ing[350]{g}{Mehl}
		\ing[1]{Pkg.}{Backpulver}
		Als nächstes Mehl und Backpulver unterrühren.
		\ing[125]{ml}{Öl}
		\ing[175]{ml}{Fanta}
		Öl und Fanta zugießen und erneut rühren.
		\newstep
		Auf dem Backblech für 25 Minuten bei 150\0C backen.
		\freeform{\Large Creme}
		\ing[600]{g}{Schlagsahne}
		\ing[3]{Pkg.}{Vanillezucker}
		\ing[3]{Pkg.}{Sahnesteif}
		Die Sahne mit Vanillezucker und Sahnesteif aufschlagen.
		\ing[500]{g}{Schmand}
		\ing[2]{Pkg.}{Vanillezucker}
		\ing{}{Früchte}
		Den Schmand mit dem Vanillezucker mischen und unter die Sahne unterheben.
		\newstep
		Die Masse auf den abgekühlten Kuchen geben und die Früchte darauf verteilen.
	\end{recipe}
	\clearpage
	\addcontentsline{toc}{section}{Lebkuchenteig (Ausstechen)}
	\begin{recipe}{Lebkuchenteig zum ausstechen}{1 Portion}{ein paar Minuten}
		\ing[600]{g}{Honig}
		Den Honig im Wasserbad flüssig werden lassen.
		\ing[200]{g}{Zucker}
		\ing[2]{}{Eier}
		\ing[200]{g}{Butter}
		Zucker mit Eier schaumig schlagen, dann weiche Butter hinzumengen.
		\ing[1200]{g}{Mehl}
		\ing[1]{Pkg.}{Backpulver}
		Mehl und Backpulver in einer zweiten Schüssel vermengen.
		\ing[1]{Pkg.}{Lebkuchengewürz}
		\ing[35]{g}{Kakao}
		Eimasse, Honig, Lebkuchengewürz und Kakao zu einem Teig verrühren.
		\newstep
		Teig ausrollen und ausstechen.
	\end{recipe}\vfill
	\addcontentsline{toc}{section}{Brownies}
	\begin{recipe}{American Brownies}{1 Blech}{\zeit{0}{40}}
		Den Backofen auf 180\0C vorheizen.
		\newstep
		\ing[400]{g}{Zartbitterschokolade}
		\ing[240]{g}{Butter}
		Schokolade und Butter zusammen schmelzen.
		\ing[240]{g}{Mehl}
		\ing[1]{TL}{Backpulver}
		\ing[\fr12]{TL}{Salz}
		Mehl mit Backpulver und Salz mischen.
		\ing[6]{}{Eier}
		\ing[420]{g}{Zucker}
		\ing[2]{TL}{Vanillezucker}
		Eier mit dem (Vanille-)Zucker schaumig schlagen und die lauwarme Schokoladenmasse hinzumengen
		\newstep
		Die Mehlmischung nach und nach hinzufügen und einrühren.
		\ing[60]{g}{Schokolade, bunt}
		\ing{einige}{gehackte Nüsse}
		Die gestückelte Schokolade und die gehackten Nüsse hinzufügen.
		\newstep
		Den Teig für etwa 20 Minuten in den auf 180\0C vorgeheizten Backofen backen.
	\end{recipe}
	\chapter{Brot}
	\ohead*{Brot}
	\addcontentsline{toc}{section}{Bananenbrot}
	\begin{recipe}{Bananenbrot}{400g/1 Brot}{\zeit{0}{45}}
		Den Backofen auf 180\0C vorheizen.
		\ing[150]{g}{Dinkelmehl}
		\ing[50]{g}{Haferflocken}
		\ing[1]{TL}{Zimtpulver}
		\ing[1]{Prise}{Salz}
		\ing[1]{Pkg.}{Backpulver}
		Mehl, Haferflocken, Zimt, Salz und Backpulver in einer Schüssel zusammenmischen.
		\ing[3]{}{Bananen}
		\ing[150]{g}{Apfelmus}
		\ing[1]{}{Ei}
		Bananen, Apfelmus und Ei in einer Schüssel mixen. Anschließend gründlich mit der Mehlmischung vermischen.
		\ing[50]{g}{gehackte Nüsse}
		Zuletzt die gehackten Nüsse untermengen.
		\newstep eine Form geben und nach Belieben mit Nüssen oder Haferflocken bestreuen. Im Backofen bei 180\0C für 35 Minuten backen.
		\newstep
		Das Brot erst nach dem Abkühlen aus der Form lösen.
	\end{recipe}
	\chapter{Kochen}
	\ohead*{Kochen}
	\addcontentsline{toc}{section}{Schilli-Con-Carne}
	\begin{recipe}{Schilli-Con-Carne}{4 Personen}{\zeit{0}{45}}
		\ing[2]{Tassen}{Reis}
		\ing[4]{Tassen}{Wasser}
		\ing[1]{Tl}{Salz}
		Reis und gesalzenes Wasser in einem Topf geben, Deckel drauf und volle Hitze! - Schon weitermachen; wenn das Wasser kocht, Platte ausschalten und Deckel drauflassen, die Restwärme reicht, um den Reis zu garen.
		\ing[500]{g}{Hackfleisch}
		\ing{}{Margarine}
		\ing{}{Knoblauch}
		Das Hackfleisch mit Knoblauch anbraten.
		\ing{}{Salz}
		\ing{}{Pfeffer}
		\ing{}{Pepperoni o.ä.}
		Das Fleisch schon einmal würzen, irgendetwas scharfes dazu tun. Herdplatte nur noch auf halbe Hitze.
		\ing[1]{Dose}{Mais}
		\ing[1]{Dose}{rote oder grüne Bohnen}
		Wenn das Fleisch gar ist, das Gemüse mit einem Sieb abtropfen und in die Pfanne schmeissen.
		\ing[2-3]{Tassen}{Wasser}
		\ing[1-2]{Tl}{Brühe}
		\ing[\fr{1}{3}]{Tube}{Tomatenmark}
		Wasser hinzufügen, Brühe einrühren und Tomatenmark hinzugeben.
		\ing[1-3]{El}{Mehl}
		Um die Sosse anzudicken nach und nach etwas Mehl hinzugeben und schnell einrühren um Klumpen zu vermeiden bis eine angenehme Konsistenz erreicht ist.
		\newstep
		Wenn der Reis alles Wasser aufgenommen hat, ist er fertig.
	\end{recipe}
	\clearpage
	\addcontentsline{toc}{section}{Kartoffelcurry mit grünen Bohnen}
	\begin{recipe}{Kartoffelcurry mit grünen Bohnen}{4 Portionen}{\zeit{0}{45}}
		\ing[1]{Kg}{Kartoffeln}
		\ing[500]{g}{grüne Bohnen}
		Kartoffeln schälen, waschen und in dicke Scheiben schneiden. Die Bohnen waschen und in ca. 2.5cm große Stücke schneiden.
		\ing[2]{EL}{Butter}
		\ing[6]{EL}{Öl}
		Butter und Öl in einer Pfanne erhitzen.
		\ing[4]{}{Chilis}
		\ing[1]{TL}{Kreuzkümmel}
		\ing[1]{TL}{Kurkuma}
		\ing[2]{Zehen}{Knoblauch}
		Die Gewürze für 30 Sekunden in die Pfanne geben.
		\ing{}{Salz}
		Die Kartoffeln hinzufügen, salzen, rühren und für 15 Minuten dünsten.
		\newstep
		Schließlich die Bohnen hinzufügen und für weitere 15 Minuten bei mittlerer Hitze dünsten.
	\end{recipe}\vfill
	\addcontentsline{toc}{section}{Burritos}
	\begin{recipe}{Burritos}{4 Portionen}{\zeit{1}{15}}
		\ing[1]{TL}{Salz}
		\ing[1]{TL}{Kreuzkümmel}
		\ing[1]{TL}{Cayennpfeffer}
		\ing[1]{TL}{Paprikapulver, edelsüß}
		\ing[\fr12]{TL}{Chiliflocken}
		\ing{}{gehackten Knoblauch}
		\ing[2]{EL}{Olivenöl}
		Für die Gewürzmarinade die Gewürze und das Öl in einer Schale verrühren und ziehen lassen.
		\ing[600]{g}{Hackfleisch}
		\ing{}{Olivenöl}
		\ing[1]{}{Zwiebel}
		\ing[2]{}{Chilischoten}
		Das Hackfleisch mit dem Öl anbraten und Zwiebeln und Chili hinzufügen.
		\newstep
		Die Gewürzmarinade hinzufügen und kurz anrösten.
		\ing[280]{g}{Mais}
		\ing[320]{g}{Kidneybohnen}
		\ing[1]{TL}{Pfeffer}
		Mais und Bohnen hinzufügen und mit Pfeffer würzen.
		\ing[8]{}{Wraps}
		\ing[200]{g}{Feta}
		\ing{etw.}{Cheddarkäse, gerieben}
		Die Wraps mit dem Hackfleisch und dem Käse füllen
	\end{recipe}\vfill
	\addcontentsline{toc}{section}{Zucchini-Gnocchi-Auflauf}
	\begin{recipe}{Zucchini - Gnocchi - Auflauf}{4 Portionen}{\zeit{1}{0}}
		\ing[4]{}{Tomaten}
		\ing[2]{}{Zucchini}
		\ing[2]{}{Zwiebeln}
		Die Gemüsen waschen und schneiden.
		\ing[etw.]{}{Öl}
		\ing[2]{Zehen}{Knoblauch}
		\ing[400]{g}{Hackfleisch}
		\ing[etw.]{}{Kräuter}
		Das Hackfleisch mit dem Knoblauch in dem Öl anbraten und die Kräuter hinzufügen.
		\ing[1]{Dose}{Tomaten}
		\ing[100]{ml}{Sahne}
		Die Dosentomaten mit der Sahne mischen und zusammen mit den Tomaten in die Pfanne geben.
		\newstep
		Den Ofen auf 180\0C vorheizen
		\newstep
		Abwechselnd Gnocchis und Hackfleischmischung in eine Auflaufform schichten.
		\newstep
		Den Auflauf für 30 Minuten Backen
		\ing[viel]{}{Käse, gerieben}
		Den Auflauf mit Käse bestreuen und weitere 15 Minuten Backen.
	\end{recipe}
\end{document}
